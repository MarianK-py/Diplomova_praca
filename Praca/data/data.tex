% !TeX spellcheck = en_EN-English

\chapter{Medical data} \label{chap:data}

Our models were developed for a database of health insurance claims maintained by the Slovak National Health Information Center (NCZI). We focused primarily on datasets related to drug prescriptions and medical procedures administered by outpatient healthcare providers.
\\

To train and verify the model, we used completely artificial data with a structure matching that of the NCZI database. The data were generated in a way that simulates realistic patient records.
\\

The data we have can be split into two categories: patient records and a mapping table.

\section{Patients records}

Patients records are split into two dataset:
\begin{itemize}
	\item Records of medical procedures from outpatient healthcare
	\item Records of prescribed drugs
\end{itemize}

In total, we have data on 173 355 patients. Each patient has at least 70 records, resulting in a dataset with 133 679 705 records overall. 

\subsection{Records of medical procedures from ambulatory health care}

Each row of this dataset corresponds to a single patient record, specifically representing one medical procedure performed on that patient. Each record consists of the following variables:

\begin{itemize}
	\item date of the procedure - date when procedure was performed
	\item code of the patient - identification code unique for the patient
	\item age of the patient - age of the patient at the time of procedure
	\item gender of the patient
	\item code of the diagnosis - identification code unique for the diagnosis for which procedure was prescribed
	\item code of the procedure - identification code unique for the medical procedure 
	\item cost of the procedure - cost associated with performing of the procedure
\end{itemize}

For our prediction model, we use most of this information. The date of the procedure, combined with the patient’s age, generates timestamp data used to chronologically order all records for a patient and serves as one dimension of the record embedding. The patient identification code enables the aggregation of all records for a single individual. Diagnosis and procedure codes are mapped to their corresponding numerical vectors and embedded into the record’s vector representation (see \ref{embedding}). Finally, cost is encoded into a cost category and serves as the target variable for training the record cost prediction models.

\subsection{Records of prescribed medicines}

Similarly to the dataset containing procedures, each row in this dataset corresponds to a single patient record, in this case representing one drug prescription for a specific patient. Each record consists of the following variables:

\begin{itemize}
	\item date of the prescription - date when drug was prescribed
	\item code of the patient - identification code unique for the patient
	\item age of the patient - age of the patient at the time of procedure
	\item gender of the patient
	\item code of the diagnosis - identification code unique for the diagnosis for which drug was prescribed
	\item code of the drug - identification code unique for the drug 
	\item cost of the drug - cost associated with performing of the procedure
\end{itemize}

The use of these variables is also similar to that for procedures, with the only difference being that, instead of encoding the procedure into the record embedding, we encode the drug.

\section{Mappings}

In addition to the patient records dataset, we utilize three mapping files to translate identification codes from the raw patient records into their corresponding embedding vectors. These mapping datasets are:

\begin{itemize}
	\item Diagnosis mapping
	\item Drug mapping
	\item Medical procedure mapping
\end{itemize}

These datasets are publicly available dimension tables maintained and used by NCZI. In general, all three mapping tables have a comparable structure consisting of three main attributes. The first is an internal identification number used by NCZI to join these mappings to patient records. The second attribute is the official code; in the case of drugs and diagnoses, this is an internationally standardized structured code, while medical procedures use a code specific to Slovakia, which unfortunately lacks any meaningful structure. The final important part is the textual description.
\\

We used these dimension tables to generate embedding vectors for drugs, diagnoses, and medical procedures. The process of this embedding is explained in Sec. \ref{embeddingImple}.
