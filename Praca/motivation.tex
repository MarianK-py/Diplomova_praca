% !TeX spellcheck = en_EN-English

%\chapter{Motivation and introduction}
\chapter*{Motivation}
% spomenut vysledky  v clanku 
% preco sme sa rozhodli spravit tuto pracu

State governments and insurance companies collect and store vast amounts of medical data, including patient histories, treatment records, prescriptions, and billing information. This data can be used for various purposes, such as detecting fraudulent activities, tracking contagious disease outbreaks, or allocating future supplies of purchased medication.
\\

Another interesting application is predicting a patient's future health, which allows for forecasts of potential diseases they may develop and the treatments they might require. Unfortunately, due to the large number of significant factors not captured in medical records, this task is almost impossible to accomplish accurately.
\\

That's why, in this study, we focus on a simpler problem: predicting the expected cost for a patient in the next year. Specifically, we aim to estimate the total anticipated costs of medications and medical procedures provided to a patient over a one-year period. In general, our approach involves two main tasks. First, we embed patient records into numerical vectors. Second, we train a model to predict each patient’s future costs based on their previous records.
\\

In the very first chapter, we briefly introduce our goal, the challenges we encountered, and the methods we used. The second chapter provides a quick overview of studies that address similar problems. The third chapter is dedicated to introducing the data we used. In the fourth chapter, we present the models and techniques employed for embedding records and for the prediction task. The fifth chapter is a brief section on the technical aspects of software design, including the programming languages and libraries we used. After that, in the sixth chapter, we describe how we implemented solutions to our two main tasks. The seventh chapter focuses on preliminary research, specifically the testing of embedding methods and various model parameters. Finally, in the eighth chapter, we discuss the results of the final model.