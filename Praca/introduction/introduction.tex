% !TeX spellcheck = en_EN-English

\chapter{Introduction}

In this chapter, we briefly introduce our goal, the challenges we encountered, and the methods we used.
\\

The main goal of our study is to develop software capable of analyzing patients’ historical records, specifically, the medications prescribed to them and the medical procedures they have undergone, in order to predict the cost category each patient will belong to in the following year. In other words, we aim to estimate the expected expenses for each patient over the next year based on their prior medical data. 
\\

To achieve the desired outcome, we faced three primary challenges. First, we needed to transform patients' historical records into numerical vectors suitable as input for machine learning models. Second, we had to simulate patients' potential futures by predicting both likely medication prescriptions and medical procedures they might undergo in the subsequent year. Finally, we required a method to calculate the expected costs of these projected treatments and aggregate them into an annual cost estimate for each patient.
\\

For record transformation (referred to as embedding later in the study), we focused on embedding medications, diagnoses, and medical procedures. We employed one of two methods depending on whether the data contained structured codes. When structured codes were available, we split the code into its hierarchical components, embedded each part individually, and then combined the results to create a comprehensive representation. For data without structured codes or meaningful organization, we utilized a machine learning-based language model to generate embeddings directly from textual descriptions.
\newpage

To generate future medical records, we evaluated several sequential data models, including LSTM (Long Short-Term Memory) networks and decoder-only Transformer architectures, as patients' future medications and procedures exhibit strong temporal dependencies on their historical records, with recent events carrying greater predictive weight.
\\

To estimate the cost category of each medical record, we primarily utilized a Multi-layer Perceptron (MLP) model. This neural network takes the numerical representation of a record (generated through our embedding process) as input and predicts its likely cost category. While the MLP served as our core approach, we also explored alternative models such as Gradient Boosting and Ridge Regression for comparative analysis.
\\

After resolving the three key challenges we integrated these components into a unified software solution. This final system sequentially processes patient histories through each stage to generate annual expense forecasts.
