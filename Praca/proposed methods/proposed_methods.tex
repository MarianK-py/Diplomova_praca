% !TeX spellcheck = en_EN-English

\chapter{Proposed Methods}
\label{propMet}

The task of predicting the future cost of a patient can be split into multiple sub-tasks, which follow each other. The sub-tasks are:

\begin{enumerate}
	\item Embed the patient’s history into numerical vectors
	\item Compute the expected number of records the patient will have in the next year
	\item Predict future records for the patient
	\item Predict the cost of each future record
	\item Compute the total cost of the patient for the next year
\end{enumerate}


% !TeX spellcheck = en_EN-English

One of sub-task for prediction of patient future costs is to embed each patient record into numerical vector. 

\section{Prediction of future number of records}

Our task is to determine how much a patient will cost in the future, specifically in the next year. Since our approach predicts future records, assigns costs to them, and sums these into a total cost, we need to estimate how many records to generate to simulate the next year. For this task, we tried two different approaches.
\\

The first approach was to predict the approximate number of records using linear regression, which takes the counts of records from previous years and predicts the next year’s count. The second approach used a stopping criterion based on timestamp information available in each input and generated record. This method stops generating records once the difference between the last timestamp from the patient’s data and the last generated timestamp surpasses a one-year threshold.
\\

Each method has its own disadvantages. In the case of linear regression, the number of records per year can vary significantly. Even if we expect an increase \cite{num_of_vis}, regression might fail to capture this trend, especially if the patient’s data includes only a few years of historical records. A potential issue with the second approach is its reliance on how well the model learned that timestamps should always increase.

% !TeX spellcheck = en_EN-English

\section{Future record prediction}
\label{record_prediction}

%This task is both the most important and the most challenging to train. Our goal is to develop a model capable of predicting a patient's potential next record based on their previous ones. Generally, this task is nearly impossible with the amount of information available, as numerous factors influence whether, when, and what new disease a patient might contract, how their current state will evolve, and what specific actions a doctor will take.
%\\

This task is both the most important and the most challenging to train. Our goal is to develop a model capable of predicting a patient’s potential next record based on their previous ones. In general, this is nearly impossible given the available information, as many other factors influence if, when, and what new disease a patient might contract, how their condition will evolve, and what actions a doctor will take.
\\

Fortunately, our ultimate goal is not to predict a patient's specific future but to estimate the likely total cost of their future records. We anticipate that even if we cannot predict exact outcomes, we can still estimate the overall cost.
\\

To achieve this prediction, we experimented with multiple models. The first three models we tested were Recurrent Neural Networks (RNNs): multi-layer Elman RNN, multi-layer Gated Recurrent Unit (GRU) RNN, and multi-layer Long Short-Term Memory (LSTM) RNN. The architecture of these models is detailed in Sec. \ref{theoryRNN}. The last model we tried was a Transformer, specifically a Decoder-only Transformer model, explained in Sec. \ref{theoryTrans}.



% !TeX spellcheck = en_EN-English

\section{Record Cost Prediction}
\label{recCostPred}

Next step in total cost prediction is to predict how much each record would cost. For this we choose standard multilayer perceptron (MLP) or in another words multi-layered fully-connected feed-forward neural network. 

\subsection{Multilayer perceptron}
\label{MLP}

Multilayer perceptron is a feed-forward neural network, so network where data flow single direction or in other words neurons don't form cycles. This network consist of fully-connected, sometimes called dense, layers with non-linear activation functions as shown on Fig. \ref{fig:mlp}, where we can see that this model can be split into three parts which are input layer which load the data, hidden layer which are trying to extract desired information using linear transformations and activation functions and finally output layer which contains final linear transformation followed by activation to output extracted information. Each layer can be described using this formula:

\begin{equation}
	\label{eqn:mlp}
	h_i = act(W_i h_{i-1} + b_i),
\end{equation}
where $h_i$ is resulting vector of i-th layer, $act()$ is a non-linear activation function, $W_i$ is weight matrix of i-th layer, $h_{i-1}$ is resulting vector from previous layer and $b_i$ is bias vector.

\begin{figure}[!h]
	\centering
	
	\includegraphics[width=0.8\textwidth]{images/MLP_arch.png}
	
	\caption{Architecture of multilayer perceptron \cite{MLParch}}
	\label{fig:mlp}
\end{figure} 
\newpage

\section{Prediction of future cost of patient}
\label{pat_fut_pred_met}

This last task is our main goal: to predict the cost of a patient in the next year. To achieve this, we utilize the results from all our previous tasks. First, we embed the patient’s records. Then, we predict the records that the patient could receive in the next year. For each predicted record, we predict its cost category. Finally, all costs are summed and transformed into a patient cost category.
