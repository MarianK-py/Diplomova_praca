% !TeX spellcheck = en_EN-English

One of sub-task for prediction of patient future is to embed patient records into numeric, this task is relatively straightforward for attributes that are already numeric like age, or for attributes which has structured code like diagnosis however in case if medical procedure we had to come up with way to embed it in a way that similar procedures would get similar embedding despite missing nicely structured code, one wy to do it is to group medical procedures into clusters and give similar embedding to procedures within cluster and dissimilar one to procedures in different clusters.
\\

For such a task Lorenzi et al. from Duke University in Durham developed novel algorithm called Predictive Hierarchical Clustering \cite{lorenzi2017predictive}. This algorithm was developed for agglomerative clustering of surgical CPT codes. This algorithm uses one-pass bottom-up approach where they utilize EHR, more precisely using 317 predictors like lab values and patients history, excluding CPT information for 3,723,252 patients and 3,132 CPT codes where each patient have one main surgical CPT code. For each CPT code then they create tree containing patients with that code. Then at each iteration, the algorithm considers merging all pairs of existing trees. To compare two trees they utilize two hypothesis, first hypothesis say that data in both trees are generated from same model, while second say data in each tree is generated from models with different parameters. Final value is weighted average of probabilities of these two hypothesis considering data in trees, where weigth is probability of first hypothesis \ref{hierClust}.

\begin{equation}
	\label{hierClust}
	p(D_k \vert T_k) = p(H_1^k)p(D_k \vert H_1^k) + (1 - p(H_1^k))p(D_i \vert T_i)p(D_j \vert T_j)
\end{equation} 

Where $D_k$ is set of data in merged tree (merged $T_i$ and $T_j$), $T_k$ is merged tree, $H_1^k$ is first hypothesis, $D_i$ and $D_j$ are data in trees $T_i$ and $T_j$. During experimental testing they compared result of their approach to clustering CPT codes into 16 clinical groups, to evaluate results they tried to use it predict additional procedures for patients in validation group using their clustering and clinical clustering, then compute area under  receiver operator curve (AUROC) and area under precision recall curve (AUROC). They found few percentage improvement in these statistics using their model compared to clinical clustering.
\\

In the end we decided to not try this approach in our work since it is not really able to distinguish whether two procedures are similar considered similar because of their real similarity or only because they commonly appear together as a determination or treatment to specific diagnosis. This distinguishment is important for since despite procedures which similarity is only that they appear together might have vastly different cost associated to them which could cause issue for prediction model.