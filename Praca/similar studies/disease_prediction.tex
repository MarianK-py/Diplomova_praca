% !TeX spellcheck = en_EN-English

From perspective of prediction of patient future one of similar studies is study called Deep Patient by Riccardo Miotto et al. \cite{miotto2016deep} where they were predicting which disease would patient have in the future based on his current state. Their input data were contained general demographic details such as age, gender and race, and common clinical descriptors such as diagnoses, medications, procedures, and lab tests. To predict future diseases they use random forest model with one-vs.-all learning. Study focus primarily on improving results of model by reducing noise in data by reducing their dimensionality. They compared standard approaches like principle component analysis, Gaussian mixture model or K-means, but main focus was approach using stack of denoising autoencoders. Model using stack of denoising autoencoders to reduce dimension showed significantly better results compared to both model using original dataset and models using other dimensionality reduction techniques.
\\

Another similar study, is study by Caballer-Tarazona er al. \cite{caballer2019predicting} in which they tried to predict future cost of the patient primarily using what they called "Aggregated Clinical Risk Group 3" computed from standardized Clinical Risk Group (CRG). This variable consist of the parts, first in one of nine grouped CRGs and second part is one of six levels of severity.