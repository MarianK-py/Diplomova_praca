% !TeX spellcheck = en_EN-English

The main goal of our study is to predict the future cost for each patient, specifically how much money will be spent on drugs and procedures for a given individual.
\\
 
One similar study in this regard is by Caballer-Tarazona et al. \cite{caballer2019predicting}, which aimed to predict patients’ future healthcare costs using an "Aggregated Clinical Risk Group 3" (ACRG3) variable derived from standardized Clinical Risk Groups (CRGs). The ACRG3 variable consists of two components: the first assigns patients to one of nine grouped CRGs, and the second categorizes them into one of six severity levels. The authors also tested whether adding demographic variables like age or sex improved model performance.
\\

To address the high proportion of zero-cost observations, they developed a two-part model:

\begin{enumerate}
	\item First part: A logit model estimating the probability of incurring non-zero costs.
	\item Second part: Either a log-linear ordinary least squares (OLS) regression or a generalized linear model (GLM) with a log link to predict expected costs for patients with positive expenditures.
\end{enumerate}

The final predicted cost was calculated as the product of the probability from the first part and the conditional expectation from the second part. Their models achieved adjusted $R^2$ values of $46.4\%$–$49.4\%$, which they noted as comparable to results from studies using alternative patient classification systems.
\\

This approach was ultimately not viable for us, as it relies on specific attributes such as CRG and patient severity level, which we did not have available.
\\

Another study focused on future patient cost prediction is by Mohammad Amin Morid et al. \cite{morid2018supervised}, which compares multiple models for predicting a patient’s future cost category using medical and pharmacy claims data.
\\

For each patient, their dataset included 21 features, all related to the amounts spent on that patient. Specifically, these features included values such as the total historical cost for the patient, the total cost in the last six months, the number of months with costs above the patient’s average, and the linear trend of cost over time.
\\

The authors aimed to predict into which of five cost categories a patient would fall in the future. To achieve this, they compared several predictive models, including Linear Regression, Decision Tree, and Artificial Neural Network (ANN). For Linear Regression, they also tested versions with regularization, such as Lasso and Ridge models. In the case of Decision Tree models, they evaluated improved ensemble methods based on decision trees, including Random Forest, Bagging, and Gradient Boosting techniques.
\\

In their testing, the most successful model turned out to be Gradient Boosting, which was able to predict the correct cost category in almost 95\% of cases. However, for the highest cost bucket-which was expected to contain only 2\% of the population-the model was successful in just 37.5\% of cases. The second and third best models were the ANN and Ridge models, both of which achieved very similar results, with around 91.5\% overall accuracy in category assignment and over 50\% correct classification in each cost category. In conclusion, they determined that Gradient Boosting generally provides the best performance for cost prediction models, while ANN offers superior results specifically for higher-cost patients.
\\

This study gave us hope that our data might contain the necessary information for accurate prediction, as most of the features in their study were computed using information about the cost of drugs or procedures-data that is present for each record in our dataset, which also contains much more detail about the causes of those costs.