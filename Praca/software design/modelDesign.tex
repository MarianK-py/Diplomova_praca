% !TeX spellcheck = en_EN-English
\newpage

\section{Model training, validation and prediction}
\label{modelDesign}

For training, validation, and using models for predictions, we primarily used \texttt{PyTorch 2.6.0}, which is an optimized tensor library for deep learning using GPUs and CPUs \cite{pytorch}. This library provided us with all the required components, such as linear, non-linear, and specialized layers, to assemble both simpler neural networks like the MLP (used for prediction of the cost category of a record) and more complex neural networks like LSTM or Transformer (used to predict future records). In addition to the building blocks for the networks themselves, PyTorch also offers other components needed for model training, such as optimizers and loss functions, with the possibility to modify them for our specific requirements, as we did for the loss function used in future record prediction.
\\

Another library used in this project was \texttt{Scikit-learn 1.2.2}, from which we utilized the linear regression model. We considered linear regression as one possible method to estimate how many future records should be generated for a patient to predict their expected costs for the next year. Additionally, Scikit-learn provided the Gradient Boosting and Ridge regression models, which we used for comparison with the MLP in the record cost category prediction task.