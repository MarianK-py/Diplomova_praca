% !TeX spellcheck = en_EN-English

\chapter{Software Design} \label{chap:softwaredesign}

This chapter is dedicated to introduce software used to create, train, validate and used machine learning model described in this thesis.
Whole code was written using Python language, more specifically in \texttt{Python 3.11.4}. We chose this language for its ease of use and wide selection of libraries for data processing and machine learning. All scripts are available at the GitHub repository \url{https://github.com/MarianK-py/diploma_thesis_code}.
\\

Whole code can be split into three parts:

\begin{enumerate}
	\item Embedding - code to add create embedding mapping files
	\item Model training and validation - code to setup, train, validate and save prediction models, has to be run once 
	\item Predictor - code to load trained models and predict future cost of inputted patients 
\end{enumerate}

Now we will introduce libraries and pre-trained models used in our code. Firstly ones used in general and then specific ones used in each part mentioned above.

\section{General}

Some of the libraries were used in all parts of code to maintain some coherence in technologies used. 
\\

To this category belong \texttt{Pandas 2.2.1} library a fast, powerful, flexible and easy to use open source data analysis and manipulation tool \cite{pandas} which was used to load, manipulate and save all datasets. Another one is \texttt{Numpy 1.26.4} library an open source project that enables numerical computing with Python \cite{numpy} which we use for computations such as random number generation, computations of mean and standard deviation for data normalization and many more.

% !TeX spellcheck = en_EN-English

\section{Embedding}
\label{embedDesign}

For embedding we used couple additional libraries since we utilized couple of algorithms and pre-trained models. More specifically libraries and specific algorithms and models we used are these: 
\\

\begin{itemize}
	\item \texttt{Scikit-learn 1.2.2} - simple and efficient tools for predictive data analysis \cite{scikitlearn},  we specifically use function to compute PCA in order to decrease dimensionality of medical procedure embedding and function K-means clustering in order to check embedding has desired property.
	
	\item \texttt{NTLK 3.8.1} - this abbreviation stands for Natural Language Toolkit, it's a library for building Python programs to work with human language data \cite{ntlk}, in our case we used tokenizer function to split description of medial procedures into tokens, in our case words. 
	
	\item \texttt{Simplemma 1.1.2} - which provides a simple and multilingual approach to look for base forms or lemmata \cite{simplemma}, we used to lemmatize our tokenized text since lemmatizer provided by this library contains also Slovak and Czech languages.  
	
	\item \texttt{SentenceTransformer 2.2.2} - go-to Python module for accessing, using, and training state-of-the-art text and image embedding models \cite{sentence_transformer}, which allowed to easily load LaBSE model from Hugging face.
	
	\item \texttt{Gensim 4.3.3} - library for topic modelling, document indexing and similarity retrieval with large corpora \cite{gensim}, which we used to load Word2vec model trained specifically for Slovak language
\end{itemize}


