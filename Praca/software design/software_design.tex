% !TeX spellcheck = en_EN-English

\chapter{Software Design} \label{chap:softwaredesign}

This chapter is dedicated to introducing the software used to create, train, validate, and use the machine learning model described in this thesis. The entire code was written in Python, more specifically in \texttt{Python 3.11.4}. We chose this language for its ease of use and wide selection of libraries for data processing and machine learning. All scripts are available in the GitHub repository at \url{https://github.com/MarianK-py/diploma_thesis_code}.
\\

The code can be divided into three parts:

\begin{enumerate}
	\item Embedding – code to create embedding mapping files
	\item Model training and validation – code to set up, train, validate, and save prediction models, this needs to be run once
	\item Predictor – code to load trained models and predict the future cost of inputted patients
\end{enumerate}

The code for model training and validation, and the code for prediction, use the same technologies, which is why they are described in a single section. Now we will introduce the libraries and pre-trained models used in our code: first those used in general, and then those specific to each part mentioned above.

\section{General}

Some of the libraries were used in all parts of the code to maintain coherence in the technologies used.
\\

In this category belongs the \texttt{Pandas 2.2.1} library, a fast, powerful, flexible, and easy-to-use open-source data analysis and manipulation tool \cite{pandas}, which was used to load, manipulate, and save all datasets. Another is the \texttt{Numpy 1.26.4} library, an open-source project that enables numerical computing with Python \cite{numpy}. We used it for computations such as random number generation, calculation of mean and standard deviation for data normalization, and many other tasks.

% !TeX spellcheck = en_EN-English

\section{Embedding}
\label{embedDesign}

For embedding we used couple additional libraries since we utilized couple of algorithms and pre-trained models. More specifically libraries and specific algorithms and models we used are these: 
\\

\begin{itemize}
	\item \texttt{Scikit-learn 1.2.2} - simple and efficient tools for predictive data analysis \cite{scikitlearn},  we specifically use function to compute PCA in order to decrease dimensionality of medical procedure embedding and function K-means clustering in order to check embedding has desired property
	
	\item \texttt{NTLK}
	
	\item \texttt{Simplemma}
	
	\item \texttt{SentenceTransformer}
	
	\item \texttt{Gensim}
\end{itemize}



Simplemma (lemmatizer)

sentence transformers (LaBSE)

gensim (word2vec)

ntlk (word tokenizer)

% !TeX spellcheck = en_EN-English
\newpage

\section{Model training, validation and prediction}
\label{modelDesign}

For training, validation, and using models for predictions, we primarily used \texttt{PyTorch 2.6.0}, which is an optimized tensor library for deep learning using GPUs and CPUs \cite{pytorch}. This library provided us with all the required components, such as linear, non-linear, and specialized layers, to assemble both simpler neural networks like the MLP (used for prediction of the cost category of a record) and more complex neural networks like LSTM or Transformer (used to predict future records). In addition to the building blocks for the networks themselves, PyTorch also offers other components needed for model training, such as optimizers and loss functions, with the possibility to modify them for our specific requirements, as we did for the loss function used in future record prediction.
\\

Another library used here was \texttt{Scikit-learn 1.2.2}, from which we used the linear regression model. We considered this as one possible way to determine how many future records we should generate for a patient to estimate their expected amount for the next year.